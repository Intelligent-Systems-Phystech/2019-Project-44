\documentclass[12pt]{article}
\usepackage{cmap}
\usepackage[T2A]{fontenc}
\usepackage[utf8]{inputenc}
\usepackage[english, russian]{babel}
\usepackage{amsmath, amsfonts,amssymb}

\textheight=24cm % высота текста
\textwidth=16cm % ширина текста
\oddsidemargin=0pt % отступ от левого края
\topmargin=-1.5cm % отступ от верхнего края
\parindent=24pt % абзацный отступ
\parskip=0pt % интервал между абзацами
\tolerance=2000 % терпимость к "жидким" строкам
\flushbottom % выравнивание высоты страниц



\begin{document}
\begin{center}
    \sc
        Московский физико-технический институт\\
        Физтех-школа прикладной математики и информатики\\
    \bf\Large
		Отзыв на выпускную квалификационную работу
\end{center}

\textbf{студента 4 курса Бучнева Валентина Сергеевича}

\textbf{Тема: <<Раннее прогнозирование достаточного объёма выборки для обобщённой линейной модели>>}

Бакалаврская работа В.\,С. Бучнева посвящена задаче раннего прогнозирования достаточного объёма выборки. Задача осложняется тем, что неизвестна структура модели. Решение задачи предлагается осуществлять в два этапа. На первом этапе определяется структура модели --- оптимальный набор признаков. На втором этапе проводится анализ зависимости функции ошибки от объёма обучающей выборки. Данная зависимость аппроксимируется некоторым параметрическим семейством функций.

В работе проведен обзор существующих методов прогнозирования достаточного объёма и описаны их недостатки. Приводится постановка задачи раннего прогнозирования достаточного объёма выборки. В экспериментальной части проведен анализ качества предложенного алгоритма на различных данных. Продемонстрирована работоспособность предложенного алгоритма.

Работа является актуальным и самостоятельным исследованием В.\,С. Бучнева. За время выполнения работы он проявил умение самостоятельно формулировать задачи и предлагать для них новые решения. Работа удовлетворяет всем требованиям, предъявляемым к бакалаврским работам, и заслуживает оценки <<отлично>>, а В.\,С. Бучнев --- присвоения квалификации бакалавра.


\vspace{1.5cm}
Научный руководитель:

А.\,И. Есенина


\end{document}